\section{Switch-Level Model}

In this appendix we discuss the basic switch-level model%
\index{switch-level model}%
{} used
by the switch-level compilers sim2ntk%
\index{sim2ntk}%
{} and ntk2exe%
\index{ntk2exe}%
{}.
Since sim2ntk is the standard sim2ntk program from the COSMOS%
\index{COSMOS}%
{}
system developed by Randy Bryant and associates at Carnegie Mellon University
and the ntk2exe program has been heavily influenced by the anamos
tool from the same system, the switch-level models are essentially the
same as the ones supported by the COSMOS system.
Furthermore, this section is heavily modeled after the user's guide
of the COSMOS system.
In particular, the structure and the examples all follow closely the
user guide.
For a more formal treatment of the underlying algorithms employed, the
reader is referred to \cite{bryant-tcad87b} and \cite{seger-imec89}.

\subsection{Circuit Model}

A switch-level circuit consists of nodes and transistors.
The ntk2exe program partitions the nodes and transistors
into subsets that are called channel-connected subnetworks.
The basic idea is to collect all nodes and transistors that are connected
through source/drain connections that do not go through the supply nodes
(power and ground).
Each such channel-connected subnetwork is then analyzed in isolation
and a behavioral model is automatically derived for the subnetwork.
The behavior of the whole circuit is then derived from the interconnection
of these behavioral models of the sub-networks.

\subsubsection{Node Model}

Each node can take on four different values $0,1, \X$, and $\top$.
Normally, $\top$---the overconstrained value---cannot be generated by
the circuit itself.
However, it can be introduced by the simulator and thus
the behavior of the network must be able to handle this value as well.
The value $\X$ is used to denote an invalid, uninitialized, or changing value.

One basic assumption in the switch-level analysis is that the complete
circuit is available.
Thus, if a node is meant to be tri-state (high-impedance), the user must
model this by attaching a dummy driver circuit.
We will return to this later.

Note also that the analysis carried out by ntk2exe takes all size
and strength effects into consideration.
Hence, the four values mentioned above suffice for modeling the
complete switch-level model with a number of different sizes and strengths etc.

There are two types of nodes:
\begin{description}
\item[input nodes]
Provide strong signals from sources external to the network.
Examples of this type of nodes are power, ground, clock and data inputs.
Note that power and ground nodes are treated specially as having fixed
logic values 1 and 0 respectively.
\item[storage nodes]
These nodes have their value determined by the switch-level analysis
and (unless they have size 0) will retain the their current values
in the absence of applied signals.
\end{description}

Each storage node is assigned a size in the set {0,...,maxnode} to
indicate (in a simplified way) its capacitance relative to other nodes with
which it may share charge\footnote{This description is accurate for the current
release of the sim2ntk and ntk2exe programs. However, a new version is
in early stages of testing that uses a partially ordered size and strength sets.
This modified version will be in the next Voss release.}
When a set of connected storage nodes is isolated from any input nodes, they
are charged to a logic state dependent only on the state(s) of the largest
node(s). Thus the value on a larger node will always override the value on
a smaller one. Many networks do not depend on charge sharing for their logical
behavior and hence can be simulated with only one node size (maxnode = 1).
In general, at most two node sizes (maxnode = 2) will suffice with high
capacitance nodes (e.g. pre-charged busses) assigned size 2 and all
others assigned size 1.

Node size 0 indicates that the node cannot retain stored charge. Whenever
such a node is isolated, its state becomes $\X$. This size is useful when
modeling static circuits. By assigning size 0 to all storage nodes, the
simulation is more efficient, and unintended uses of dynamic memory
can be detected.

The symbolic analyzer ntk2exe attempts to identify and eliminate storage
nodes that serve only as interconnections between transistor sources and
drains in the circuit.
It retains any node that it considers ``interesting,'' i.e., whose state
affects circuit operation.
Interesting nodes include those that act as the gates of transistors,
as inputs to functional blocks, or as sources of stored charge to other
interesting nodes.
Sometimes a node whose state is not critical to circuit operation,
however, may be of interest to the simulator user.
The user must take steps to prevent ntk2exe from
eliminating these nodes, by identifying them as ``visible.''
A node can be so identified by an attribute in the .sim file, or with
a command line option to ntk2exe.

\subsubsection{Transistor Model}

A transistor is a three terminal device with node connections gate,
source, and drain.
Normally, there is no distinction between source and drain connections--the
transistor is a symmetric, bidirectional device.
However, transistors can be specified to operate unidirectionally to overcome
limitations of the network model.
That is, a transistor can be forced to pass information only from its
source to its drain, or vice-versa.
Unidirectional transistors are required only rarely in such circuits as
sense amplifiers and pass transistor exclusive-or circuits.
Excessive use of unidirectional transistors can cause the simulator
to overlook serious design errors.
Any circuit simulated with unidirectional transistors
should be thoroughly analyzed with a circuit simulator such as SPICE.

Each transistor has a strength in the set {1,...,maxtran}.
The strength of a transistor indicates (in a simplified way) its
conductance when turned on relative to other transistors which may form
part of a ratioed path.
When there is at least one path of conducting transistors to a storage node
from some input node(s), the node is driven to a logic state
dependent only on the strongest path(s), where the strength of a path
equals the minimum transistor strength in the path.
Thus, a stronger signal will always override a weaker one.
Most CMOS circuits do not involve ratioing, and hence can be simulated with one
transistor strength (maxtran = 1).
However, circuits involving multiple degrees of ratioing may require more
strengths.
Ntk2exe will utilize as many node sizes and transistor strengths as are used
in the network file with the limitation that maxnode + maxtran < 16.

The simulator models three types of transistors: n-type, p-type, and depletion.
A transistor acts as a switch between source and drain controlled by
the state of its gate node as follows: When a transistor is in an ``unknown''
state it forms a conductance of unknown value between (inclusively) its
conductance when ``open'' (i.e. 0.0) and when ``closed''.
The simulator models these transistors in such a way that any node
with state sensitive to their actual conductances is set to $\X$.

\begin{table}[htbp]
\begin{center}
\begin{tabular}{|c|c|c|c|} \hline
gate  & n-type &p-type &depletion \\ \hline
0     & open   &closed &closed \\ \hline
1     & closed &open   &closed \\ \hline
$\X$     & unknown&unknown&closed \\ \hline
\end{tabular}
\label{TransFun}
\caption{Transistor State as a Function of Gate Node State}
\end{center}
\end{table}

Normally, transistor switching is simulated with a unit delay model.
That is, one simulation time unit elapses between when the gate node of
a transistor changes state, and the subcircuit containing the source
and drain nodes of the transistor is evaluated.
However, a transistor can be specified to have zero delay, meaning
that the subcircuit will be evaluated immediately.
The implications of the transistor delay are discussed more in
Section~\ref{TimingModel}.
Zero delay transistors are required only in rare cases
to correct for the effects of circuit delay sensitivities.
They can also be used to speed up the simulation, by
creating rank-ordered evaluation of the circuit components.

\subsubsection{Circuit Partitioning}

\FIG{ntkFiniteStateMach}{Finite State Behavior of a Circuit Module.}{ntkFiniteStateMach}

Ntk2exe partitions the initial circuit description into a set of modules.
Each module corresponds to a transistor subnetwork.
A subnetwork consists of a set of storage nodes connected by sources and drains
of transistors, along with all transistors for which these nodes are
sources or drains.
Observe that an input node is not in any subnetwork, but a transistor for
which it is a source (or drain) will be in the subnetwork containing the
drain (or source) storage node.

The behavior of a module is described by a procedure generated
automatically for the corresponding subnetwork.

As illustrated in \fig{ntkFiniteStateMach}, each module behaves like
a finite state machine, computing new values for the results as a function
of the old values on the results and inputs.
The box labeled with D in the figure represent the delays of the
various nodes and is always at least one time unit long.
We will return to this shortly.

The partitioned circuit must obey the following rules:
\begin{enumerate}
\item
A node can be a result connection of at most one module.
\item
There can be no zero-delay cycles, i.e., every cycle in the set
of interconnected modules must be broken by at least one unit delay.
\end{enumerate}

\subsubsection{Timing Model}
\label{TimingModel}.

\label{DelayAnnotation}
The simulation program is designed primarily for simulating clocked
systems, where there is a well defined timing regime for when signals
should be stable and when outputs are expected.
Although the sim2ntk and ntk2exe is only able to derive unit/zero
delay models, the user can back annotate the generated .exe file
by creating a file with the same name as the .exe file, but with a
.del suffix.
In this file, the user can list node names and bounds on the rise
and fall delays of the nodes.
An entry for a node in the .del file is of the format:
\begin{hol}
node\_name |min\_rise\_delay max\_rise\_delay min\_fall\_delay max\_fall\_delay|
\end{hol}
where the rise and fall delays must be integer multiples of the basic time unit.

The basic Fl system is able to analyze circuits using several timing models:
\begin{enumerate}
\item
unit delay model
\item
nominal delay
\item
minimum delay
\item
maximum delay
\item
bounded delay
\end{enumerate}

In the presence of a .del file and the appropriate delay model chosen
in the .vossrc file, the STE command in Voss will carry out the
chosen delay analysis.

\subsection{Circuit Examples}

In the next release of this document we will provide some
examples to illustrate the switch-level model.
