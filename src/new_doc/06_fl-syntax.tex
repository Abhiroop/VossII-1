\section{Syntax Summary}

This is a (somewhat edited) version of the parser for FL. Since FL
is still evolving, the actual
syntax accepted by the program may differ slightly from this one. However,
I have tried to make
the grammar as close as possible to the parser in Voss 1.8b.

\begin{hol}
/* Top level commands */

pgm             : pgm ';' stmt
                | stmt
                ;

stmt            : expr
                | decl
                | type_decl
                | 'install_print_function' expr
                | '{' simple_type '}' decl
                | 'print_all_fns'
                | 'postfix' var_or_infix
                | 'nonfix' var_or_infix
                | 'binder' var_or_infix
                | 'infix' NUMBER var_or_infix
                | 'infixr' NUMBER var_or_infix
                | 'begin_abstype'
                | 'end_abstype' var_or_infix_list
                | /* Empty */
                ;

var_or_infix    : VART 
                | INFIX_VAR
                | INFIXR_VAR
                | POSTFIX_VAR 
                | BINDER_VAR 
                ;

var_or_infix_list : var_list var_or_infix
                  | var_or_infix
                  ;
\end{hol}
\begin{hol}
/* Declarations */
decl            : 'let' fn_defs
                | 'letrec' fn_defs
                ;

fn_defs         : fn_def '/\.' fn_defs
                | fn_def
                ;
fn_def          : var_or_infix lhs_expr_list
                | '(' var_or_infix lhs_expr_list ')'
                ;
lhs_expr_list   : expr1 lhs_expr_list
                | '=' top_expr
                ;
\end{hol}
\begin{hol}
/* Type declarations */
type_decl       : 'lettype' type_name '=' type_expr_list
                | 'new_type_abbrev' var_or_infix '=' simple_type
                ;
type_name       : type_name ',' var_or_infix
                | var_or_infix
                ;
type_expr_list  : type_expr_list ',' type_expr
                | type_expr
                ;
type_expr       : type_expr '|' type
                | type
                ;
type            : var_or_infix type_list
                ;

type_list       : type_list simple_type
                | /* Empty */
                ;
simple_type     : var_or_infix
                | simple_type '->' simple_type
                | simple_type '#' simple_type
                | simple_type 'list'
                | '(' simple_type ')'
                ;
\end{hol}
\begin{hol}
/* Expressions */
top_expr        : expr ;

expr            : decl 'in' expr
                | 'val' expr1 '=' expr 'in' expr
                | '\.' expr1 '.' expr
                | BINDER_VAR expr1 '.' expr
                | expr '=>' expr '|' expr
                | expr POSTFIX_VAR
                | expr INFIX_VAR expr
                | expr INFIXR_VAR\ expr
                | expr ',' expr
                | expr '=' expr
                | expr05
                ;
expr05          : expr05 expr1
                | expr1
                ;

expr1           : '[' expr_list ']'
                | '[' ']'
                | '(' expr ')'
                | expr2
                ;

expr_list       : rev_expr_list
                ;

rev_expr_list   : rev_expr_list ',' expr
                | expr
                ;

expr2           : VART
                | NUMBER
                | '[]'
                | '" ... "'
                | 'quit'
                ;
\end{hol}

\subsection{Reserved Words in FL}

The following list contains all identifiers that are currently defined
in FL. This list will likely change in future releases.
\begin{quote}
{\bf
begin\_abstype
binder
end\_abstype
forall\_last
HOL\_DEF
HOL\_EXPR
infix
infixr
in
install\_print\_function
let
letrec
lettype
HOLlettype
list
new\_type\_abbrev
nonfix
overload
postfix
val
quit}
\end{quote}
In addition, the following symbols are also defined and cannot be
redefined:
\begin{quote}
{\bf
\verb@=>@
\verb@->@
\verb@|@
=
\verb@#@
\verb@/\@
\verb@\@
::
}
\end{quote}

\newpage
\section{Buit-in Functions and Commands}

The following subsections contain all built-in functions and
top-level comamands in FL.
Most of them are described in the tutorial section, but this section
should be viewed as the reference section.

\subsection{Functions}


Note that all these functions can be redefined by the user or
by library files, and thus the table is only valid for ``raw'' fl.
The functions are given in alphabetical order.

\begin{description}

\index{bool2str}
\FLfunction{bool2str}{bool\funarrow string}{bool2str f}{%
Convert the Boolean function {\tt f} to a Boolean expression
and return a string that prints out this expression.
Note that the expression can be truncated if the size
of the Boolean expression exceeds some threshold.
For more details, see the section on printing
Boolean functions on page~\protect\ref{printBooleans}.}

\index{bdd\_size}
\FLfunction{bdd\_size}{(bool list)\funarrow int}{bdd\_size [f1,f2,...,fn]}{%
Return the number of OBDD nodes needed to represent the functions
f1, f2, ..., fn.
Note that bdd\_size[f1]+bdd\_size[f2] is usually significant
larger than bdd\_size[f1,f2], due to sharing in the OBDD structure.}

\index{bdd\_load}
\FLfunction{bdd\_load}{string\funarrow (bool list)}{bdd\_load FileName}{%
Read in the OBDD functions from file FileName.
This fucntion assumes the OBDDs were saved in the format
used by bdd\_save.}

\index{bdd\_reorder}
\FLfunction{bdd\_reorder}{int\funarrow bool}{bdd\_reorder n}{%
Invoke the dynamic variable re-ordering routine in order
to reduce the size of the OBDDs.
The parameter $n$ denote the number of times the
re-ordering should be done.
}

\index{bdd\_save}
\FLfunction{bdd\_save}{string\funarrow ((bool list)\funarrow bool)}{%
bdd\_save FileName [f1,f2,...,fn]}{%
Save the OBDDs rooted at f1, f2, ..., fn.
The format of the saved file is in itself a valid fl program
and thus is fairly self-explanatory.
}

\index{chr}
\FLfunction{chr}{int\funarrow string}{chr n}{%
Return the string (of length 1) of the character corresponding
to the ASCII code n.
If n is less than 0 or greater than 127, the function fails with
a ranger error message.
}


\index{depends}
\FLfunction{depends}{(bool list)\funarrow (string list)}{depends [f1, f2, ..., fn]}{%
Return the list of names of Boolean variables that the functions
f1, f2, ..., fn depends on.
Note that the order of the variables in the string correspond to the
OBDD variable ordering.
}

\index{error}
\FLfunction{error}{string\funarrow failure}{error msg}{%
Raise an exception with error message msg.
}

\index{empty}
\FLfunction{empty}{(*list)\funarrow bool}{empty l}{%
Return T (true) if list l is empty. Otherwise return F (false).
%%% CJS: Is this one needed anymore given pattern matching?
}

\index{explode}
\FLfunction{explode}{string\funarrow (string list)}{explode s}{%
Convert the string s into a list of strings of length 1.
}

\index{eval}
\FLfunction{eval}{string\funarrow bool)}{eval s}{%
Eval writes out the string s on a temporary file, redirects
FL's standard input to this file and returns T if this was succsessful.
Otherwise it return F and leaves the standard input as before.
Note that this function is most useful at the top level
where it can be used to implement reference variables.}

\index{fanin}
\FLfunction{fanin}{fsm\funarrow (string\funarrow (string list))}{fanin ckt n}{%
For fsm object ckt, return the list of node names that the next state
function for node s depends on.
}

\index{fanout}
\FLfunction{fanout}{fsm\funarrow (string\funarrow (string list))}{fanout ckt n}{%
For fsm object ckt, return the list of node names whose next state
functions depend on the value of node s.
Note that fanout uses the topology of ckt to detremine fanout.
Thus, it is possible that the function is conservative and returns
node names that, functionally speaking, are not in the fanout set of node s.
}

\index{STE}
\FLfunction{STE}{string\funarrow fsm\funarrow (fourtuple list) \funarrow (fivetuple list)\funarrow (fivetuple list)\funarrow ((string\verb@\#@int\verb@\#@int) list)\funarrow bool\\ where fourtuple is bool\verb@\#@string\verb@\#@int\verb@\#@int and fivetuple is bool\verb@\#@string\verb@\#@bool\verb@\#@int\verb@\#@int}{STE options ckt ant cons tracenodes}{%
Perform symbolic trajectory evaluation on the circuit ckt.
The options is a string that give various options to the
symbolic trajectory evaluator.
Currently recognized options are:
\begin{description}
\item[-a] Abort the verification at the first antecedent or consequent failure.
If the verification is aborted, STE will return a Boolean function that
gives {\em the condition for this failure to manifest itself}.
Note that this is contrary to STE's usual behavior which is to return
the Boolean function that gives the conditions for the verification to
succeed.
\item [-m n]
Abort the verification after reaching time n.
\item [-i]
Allow antecedent failures.
In other words, compute a straight implication.
The normal behavior of the verification process is to disallow
antecedent failures.
Thus the default verification condition is both to check that every
trajectory the circuit can go thorough that is consistent with the
antecedent is also consistent with the consequent, and that there
is at least one (real) circuit trajectory that is consistent
with the antecedent.
\item[-w]
Do not print out warning messages.
\item[-t s]
In addition to printing out trace messages on stderr, also send
the trace events in Postscript format to the file s.
By previewing or printing out the file the user gets a waveform diagram for
the traced signals.
\item[-T s]
Same as -t, but generate Postscript code in landscape mode.
\end{description}
For more details about symbolic trajectory evaluation, see
Section~\protect\ref{SymbTrajSection}.
}

\index{fst}
\FLfunction{fst}{(*\verb@\#@**)\funarrow *}{fst obj}{Return the first part of a pair.}

\index{get\_node\_val}
\FLfunction{get\_node\_val}{fsm\funarrow (string\funarrow (bool\verb@\#@bool))}{get\_node\_val fsm nd}{%
Return the current value on node nd in the model structure fsm.
This function is mostly useful after evaluating an STE command
using the -a (abort) or -m (maximum number of steps) options.
The pairs of Boolean functions returned represent the high
and low values respectively.
As usual, the encoding is 0=(F,T), 1=(T,F), X=(T,T), and top=(F,F).}

\index{get\_delays}
\FLfunction{get\_delays}{fsm\funarrow (string\funarrow ((int\verb@\#@int)\verb@\#@(int\verb@\#@int)))}{%
get\_delays fsm nd}{%
Return the tuple
((min-rise-delay, max-rise-delay), (min-fall-delay, max-fall-delay))
for the node nd in the model structure fsm.}

\index{hd}
\FLfunction{hd}{(* list)\funarrow *}{hd l}{Return the head (first element) of a
list.}
%%%CJS: Should hd be removed given pattern matching?

\index{implode}
\FLfunction{implode}{(string list)\funarrow string}{implode sl}{%
Concatenate the strings in sl and return the resulting string.
Note that there is a limit on how big a string can be.
Currently this limit is around 16,000 character.
Note also, that currently the strings are not garbage collected.
However, all strings are made unique internally and thus only
the total number of distinct strings determines the memory usage
for the strings.}
%%%CSJ: Do I actually need to say this here?

\index{int2str}
\FLfunction{int2str}{int\funarrow string}{int2str i}{Returns a string version
of the integer i.
If i is negative the string will begin with a minus sign.}

\index{is\_stable}
\FLfunction{is\_stable}{fsm\funarrow bool}{is\_stable ckt}{%
Returns true if the circuit ckt is currently stable, i.e., no
the value on each node equals its excitation.
This function is mostly useful after evaluating an STE command
using the -a (abort) or -m (maximum number of steps) options.}

\index{load}
\FLfunction{load}{string\funarrow bool}{load fn}{%
Re-direct standard input to read from the file fn until the
file is exhausted at which point it is restored to its current
source.
Returns T (true) if the file can be found.
Returns F otherwise.
If an error occurs during reading a file, all nested
reads are terminated and the system returns to the
user level prompt.}

\index{load\_exe}
\FLfunction{load\_exe}{string\funarrow fsm}{load\_exe fn}{%
Loads an fsm object described in the .exe format
stored in the file named fn.
Note that node delays can be provided in a companion
file (with same name but with suffix .del).
}
%%%CJS: Does it now use the same search path as for .vossrc???

\index{make\_fsm}
\FLfunction{make\_fsm}{Set\funarrow fsm}{make\_fsm cl}{%
Converts a next state function defined as the concrete data type Set
to the internal fsm object.
For more details on the Set data type, see Section~\protect\ref{makeSTEsection}.
}

\index{NOT}
\FLfunction{NOT}{bool\funarrow bool}{NOT f}{Returns the negation of the
Boolean function f.}

\index{nodes}
\FLfunction{nodes}{fsm\funarrow ((string list) list)}{nodes fsm}{%
Returns the list of nodes that the model structure fsm contains.
For each node, the list of names (including aliases) are returned.}

\index{ord}
\FLfunction{ord}{string\funarrow int}{ord s}{%
Returns the ASCII code for the first character in the string.
The function fails with a length constraint message if it is not
given a string of length 1.}

\index{print}
\FLfunction{print}{string\funarrow string}{print s}{%
Prints out the string s on standard output and returns
an emtpy string.
Note that due to FL's lazy semantics, constants will
usually only be printed out once.
For example,
\vspace{-5mm}
\begin{hol}
let ans = print "OK"; ans; ans; ans;
\end{hol}
will only print out "OK" once, rather than three times.
The usual work-around is to catenate constant strings
with some variable string and then print out the final string.}

\index{print\_fsm}
\FLfunction{print\_fsm}{fsm\funarrow string}{print\_fsm fsm}{%
Prints out the internal representation of a model structure.
For more details of the format, see Sections~\protect\ref{fsmStructureSection}. }

\index{profile}
\FLfunction{profile}{(bool list)\funarrow (int list)}{profile l}{%
Returns the widths of the OBDD multigraph indicated by the
list of Boolean functions in l.
Note that this routines correctly accounts for sharing.
If individual function sizes are needed, call profile
with singelton lists.
User's rarely call this function directly, but rather calls the
function bdd\_profile defined in the defaults.fl file which
prints out a graphical profile of the OBDD multigraph.}

\index{rvariable}
\FLfunction{rvariable}{string\funarrow bool}{rvariable s}{
\label{rvariable}
Returns a random Boolean value.
Note that there is only one value associated with each variable name s.
Thus, (rvariable "a") == (rvariable "b") always holds.}

%%%CJS: Should these be removed?
%\index{relprod\_thereis}
%\FLfunction{relprod\_thereis}{(bool list)\funarrow (int list)}{profile l}{ }
%\index{relprod\_forall}
%\FLfunction{relprod\_forall}{(bool list)\funarrow (int list)}{profile l}{ }

\index{quant\_forall}
\FLfunction{quant\_forall}{bool\funarrow (bool\funarrow bool)}{quant\_forall vf f}{%
Returns the result of universally quantifying out all variables
occuring in the Boolean function vf from the Boolean function f. }

\index{quant\_thereis}
\FLfunction{quant\_thereis}{bool\funarrow (bool\funarrow bool)}{quant\_thereis vf f}{%
Returns the result of existentially quantifying out all variables
occuring in the Boolean function vf from the Boolean function f. }

\index{snd}
\FLfunction{snd}{(*\verb@\#@**)\funarrow **}{snd obj}{Returns the second part of a pair.}

\index{save\_exe}
\FLfunction{save\_exe}{fsm\funarrow (string\funarrow bool)}{save\_exe fsm s}{%
Save the model structure fsm in the exe format in file s.
Returns true if it is successful and false otherwise.}

%%%CJS: Should these be removed?
%\index{string\_hd}
%\FLfunction{string\_hd}{string\funarrow string}{profile l}{ }
%\index{string\_tl}
%\FLfunction{string\_tl}{string\funarrow string}{profile l}{ }

\index{substitute}
\FLfunction{substitute}{((string\verb@\#@bool) list)\funarrow (bool\funarrow bool)}{substitute sl f}{%
Simultaneously perform the substitutions given in the substitution list
sl in the Boolean function f.
A substitution in the list is a pair--variable to be substituted and
boolean function to be substituted in.
Formally, given the substitution list $\mbox{sl}=[(v_1,f_1),\ldots,(v_n,f_n)]$,
we have $\mbox{substitute sl f} =
\exists v_1. \ldots \exists v_n.\; ((v_1=f_1) \wedge \ldots \wedge(v_n=f_n) \wedge f)$.
}

\index{system}
\FLfunction{system}{string\funarrow int}{system s}{%
System causes the string s to be given to sh(1) as input as
if the string had been typed as a command at a terminal.
The current process waits until the shell has completed, then
returns the exit status of the shell.}

%%%CJS: Should these be removed?
%\index{SUC}
%\FLfunction{SUC}{int\funarrow int}{profile l}{ }

\index{tl}
\FLfunction{tl}{(* list)\funarrow (* list)}{tl l}{%
Returns the tail (all but the first element) of a list.
Fails and raises an exception on the empty list.}
%%%CJS: Should tl be removed given pattern matching?

\index{time}
\FLfunction{time}{*\funarrow (*\verb@\#@string)}{time e}{%
Evaluate e and return the pair (e,t), where t is a string
containing the usertime used to evaluate e.
Note that the laziness of FL may make consecutive invocations
require vastly different amount of time.
On HP machines, time returns the wall time. This is a ``feature''
that is likely to be removed soon. }

\index{trace}
\FLfunction{trace}{string\funarrow fsm\funarrow (fourtuple list) \funarrow (fivetuple list)\funarrow (fivetuple list)\funarrow ((string\verb@\#@int\verb@\#@int) list)\funarrow bool\\ where fourtuple is bool\verb@\#@string\verb@\#@int\verb@\#@int and fivetuple is bool\verb@\#@string\verb@\#@bool\verb@\#@int\verb@\#@int}{trace options ckt ant cons tracenodes}{%
Performs the same operations as STE (with the same arguments), but
returns a list of node changes.
For each node in the trace list and for each time t that node change
value,  a tripple (t,H,L) is added to a list that is returned by
the trace command.
The H and L are the boolean functions for the high and low
value on a node using the standard encoding, i.e., 0 is (F,T),
1 is (T,F), X is (T,T) and overconstrained is (F,F).
Note that the command is eager and thus the whole trace is
computed before the list is returned.
As a result, the command can be very expensive to evaluate since
there may be a large number of distinct Boolean functions
generated that must be rememberd (and thus not garbage collected). }

\index{variable}
\FLfunction{variable}{string\funarrow bool}{variable s}{%
Returns the Boolean function that is T if the Boolean variable
s is T and F otherwise.}

\index{var\_order}
\FLfunction{var\_order}{(string list)\funarrow bool}{var\_order l}{%
Re-orders the Boolean variables so that the variables
listed in l will be the first variables in the OBDD ordering.}

\index{AND}
\FLfunction{AND}{bool\funarrow (bool\funarrow bool)}{f AND g}{%
Computes the AND of two Boolean functions.}

\index{OR}
\FLfunction{OR}{bool\funarrow (bool\funarrow bool)}{f OR g}{%
Computes the OR of two Boolean functions.}

\index{XOR}
\FLfunction{XOR}{bool\funarrow (bool\funarrow bool)}{f XOR g}
\FLfunction{\%}{int\funarrow (int\funarrow int)}{f\%g}{%
Computes f MOD  g.
Note that FL uses arbitrary precision integers.}

\index{*}
\FLfunction{*}{int\funarrow (int\funarrow int)}{f*g}{%
Computes f times g.
Note that FL uses arbitrary precision integers.  }

\index{/}
\FLfunction{/}{int\funarrow (int\funarrow int)}{f/g}{%
Computes f divided by g.
Note that FL uses arbitrary precision integers.  }

\index{+}
\FLfunction{+}{int\funarrow (int\funarrow int)}{f+g}{%
Computes f plus g.
Note that FL uses arbitrary precision integers.  }

\index{-}
\FLfunction{-}{int\funarrow (int\funarrow int)}{f-g}{%
Computes f minus g.
Note that FL uses arbitrary precision integers.}

\index{:}
\FLfunction{:}{*\funarrow (* list)\funarrow (* list)}{e:l}{%
Inserts element e as the first element in the list l.}


%\index{\verb@^@}
\FLfunction{\protect\mbox{\protect\^{} }}{string\funarrow (string\funarrow string)}{f \protect\mbox{\protect\^{} } s}{%
Catenates strings f and s and return the result.
Note that the maximum string length in fl is currently
about 16,000 characters and strings longer than that are
simply truncated (after a warning message is printed out
to standard error).}

\index{$<$}
\FLfunction{$<$}{int\funarrow (int\funarrow bool)}{f$<$g}{%
Returns T if f$<$g and F otherwise.}

\index{$<$=}
\FLfunction{$<$=}{int\funarrow (int\funarrow bool)}{f$<$=g}{%
Returns T if f$<=$g and F otherwise.}

\index{$>$}
\FLfunction{$>$}{int\funarrow (int\funarrow bool)}{f$>$g}{%
Returns T if f$>$g and F otherwise.}

\index{$>$=}
\FLfunction{$>$=}{int\funarrow (int\funarrow bool)}{f$>$=g}{%
Returns T if f$>=$g and F otherwise.}

\index{=}
\FLfunction{=}{*\funarrow (*\funarrow bool)}{f=g}{%
If f and g are Boolean functions, f=g returns the exclusive
NOR of the two functions.
If f and g are any other functions, the result is T if the two
functions have the same name and F otherwise.
For other values, f=g returns T if f and g are structurally
equal and F otherwise.}

\index{==}
\FLfunction{==}{*\funarrow (*\funarrow bool)}{f==g}{%
If f and g are Boolean functions, f==g returns T if the two
functions are equal and F otherwise.
For other values, f==g and f=g behaves identically.}

\index{\!=}
\FLfunction{!=}{*\funarrow (*\funarrow bool)}{f!=g}{%
If f and g are Boolean functions, f!=g returns the exclusive
OR of the two functions.
If f and g are any other functions, the result is F if the two
functions have the same name and T otherwise.
For other values, f!=g returns F if f and g are structurally
equal and T otherwise.}

\index{catch}
\FLfunction{catch}{*\funarrow *\funarrow *}{e catch f}{%
Forces a complete evaluation of e.
If e results in an exception f is returned.
Otherwise the result of evaluating e is returned.}

\index{seq}
\FLfunction{seq}{*\funarrow **\funarrow **}{e seq f}{%
Evaluates first e and then f and returns the result of evaluating f.}

\end{description}

\subsection{Top-level commands}

The following set of functions/commands are only available at the
top level, i.e., directly after a prompt (if the session had been done
interactively).
Most of these commands modify the way FL parses input or presents output
and are meant to make it easier to embed other languages in FL.
As with all such user-defined changes to the parser, havoc can easily
break out. 
Thus, we strongly sugest that careful consideration be given
to which operators/functions that will be overloaded, made infix or
postfix, and which types gets userdefined print routines.
\begin{description}
\index{install\_print\_function}
\FLfunction{install\_print\_function}{(*\funarrow string)\funarrow ()}{install\_print\_function f}{%
Install\_print\_function takes a function f of type *\funarrow string,
where * is a concrete type (defined using lettype) and installs
this function as the pretty printer for objects of type *.
Thus whenever an object of type * gets evaluated at the top level, fl
will automatically call the function f and print out the resulting string.
Note that only concrete types can have pretty printer functions
associated with them.
}
\index{binder}
\FLfunction{binder}{((*\funarrow **)\funarrow string\funarrow **)\funarrow ()}{binder f}{\mbox{}}
Binder takes a function and makes it into a binder, i.e., an object that
can introduce a new bound variable in an expression.
Note that the type of the function must be (*\funarrow **)\funarrow string\funarrow **,
since the first argument of a binder function will be a lambda expression
and the second argument will be a string with the name of the bound variable.
Thus, if a function f has been declared as a binder, then
f x.E will be parsed as f (\protect\verb!\! x.E) "x".
\index{postfix}
\FLfunction{postfix}{(*\funarrow **)\funarrow ()}{postfix f}{%
Postfix takes a name of a function and modifies the parser so
that the function will be parsed as a postfix operator.
Postfix operators have higher precedence than infix, infixr, and
prefix functions.
}
\index{nonfix}
\FLfunction{nonfix}{(*\funarrow **)\funarrow ()}{nonfix f}{%
Nonfix takes a name of a function and removes the special parsing of
the function. If the function is not declared as a binder, postfix,
infix or infixr operator, nonfix is (essentially) a no-op.
}
\index{infix}
\FLfunction{infix}{(int\funarrow *\funarrow **)\funarrow ()}{infix p f}{%
Infix takes a number p between 0 and 9 and a function or operator f.
The command causes the parser to view f as an infix operator with
precedence p that associates to the left.
}
\index{infixr}
\FLfunction{infixr}{(int\funarrow *\funarrow **)\funarrow ()}{infixr p f}{%
Infixr takes a number p between 0 and 9 and a function or operator f.
The command causes the parser to view f as an infixr operator with
precedence p that associates to the right.
}
\FLcommand{overload}{overload s f1 f2 f3 ... fn}{%
Overload takes a name s and a list of names (f1 .. fn)
(possibly decorated with type) of functions or operators.
From here on, the FL parser will overload the name s
to denote anyone of the functions.
Note that the oveloading in FL is somewhat restricted, since only
the argument types are used in resolving the overloading.
As a result, it is often necessary to type the arguments
to a function explicitly in order for FL to succeed in
resolving the function requested.
As usual, with overloading, it is easy to create programs that
appears to be correct, but that denotes something completely
different.
Thus, we strongly suggest the overloading of operators should be
used sparingly.
}
\FLcommand{begin\_abs\_type}{begin\_abs\_type;}{%
Begin\_adt starts the definition of a (possibly nested)
abstract data type.
}
\FLcommand{end\_abs\_type}{end\_abs\_type f1 f2 ... fn;}{%
End\_abs\_type takes a list of functions and/or type constructors to
be exported from the abstract data type.
Thus of all the functions and operators defined inside the abstract
data type, only the ones listed after end\_abs\_type will be visible
to other functions.
}
\FLcommand{set\_prompt}{set\_prompt s;}{%
Change the interactive prompt to the string s.
}
\end{description}

