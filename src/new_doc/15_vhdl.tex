\section{VHDL Support}
\label{VHDL}

The support for behavioral and structural VHDL%
\index{VHDL}%
{} is through a translation
program derived from the Alliance%
\index{Alliance}%
{} 1.1 distribution. 
Since the Alliance 1.1 tools are distributed under the Free Software
Foundations license agreement, the source to this translator is
available to whoever wants it.

The VHDL subset supported is fully compatible with the IEEE VHDL standard
ref. 1076 (1987).
Hopefully this means that any program that is acceptable to the convert2fl%
\index{convert2fl}%
{}
program, will also be acceptable to commercial synthesis and simulation tools. 
However, I don't give any guarantees.
Below we outline the main restrictions of the VHDL subset we support.

A VHDL description of a circuit consists of two parts:
the external view and the internal view.
The external view defines the name of the circuit and its interface.
The interface is a list of ports.
Each port is specified by its name, mode, type, possible constraint, and its
kind.
The mode of a port depends only on the manner the port is used inside the
circuit (in the internal view of the circuit).
If the value is to be read in the view of the description, the port must
be declared with the mode {\em in}.
If the value is to be written by the internal view, the port must
be declared with the mode {\em out}.
If the internal view will both read and write to the port, the mode of the
port must be {\em inout}.

Only {\em structural} and {\em behavioral data flow} representations
are supported as internal view.
Furthermore, it is not possible to mix behavioral and structural
descriptions of a single entity.
The convert2fl program requires also that each entity is contained in
a file with the same name as the entity (lower case letters only!).
If the description of the entity is structural, the suffix must be .vst,
whereas if the description if behavioral, the suffix should be .vbe.

A typical VHDL description of a circuit will consist of a collection
of files creating a structural hierarchy of .vst files with behavioral
.vbe descriptions of the leaf nodes in the hierarchy.
Note that the convert2fl program will first search for a behavioral
description of an entity.
Only if this fails, will the program look for a structural description
of the entity\footnote{If convert2fl fails to find either a .vbe or .vst
description, it looks for an EDIF description (.edi suffix)}.

\subsection{Types Supported}

The following set of predefined types has been defined.
No other user-defined types are currently supported.
\begin{description}
\item[bit]
The predefined standard bit type ('0' or '1').
In the Voss system, this type is monotonically extended to the
quaternary domain.
\item[bit\_vector]
An array of bits.
\item[mux\_bit]
A resolved subtype of bit using the {\em mux} resolution function.
This function computes the greatest lower bound of the actively
driven signals.
If all drivers are disconnected, the value of the signal will be $\X$.
Note that signal of type mux\_bit must be declared with the kind {\em bus}.
\item[mux\_vector]
An array of mux\_bits.
\item[reg\_bit]
A resolved subtype of bit using the {\em reg} resolution function.
This function computes the greatest lower bound of the actively
driven signals.
If all drivers are disconnected, the value of
the signal will retain its old value.
Note that signal of type reg\_bit must be declared
with the kind {\em register}.
\item[reg\_vector]
An array of reg\_bits.
\end{description}

\subsection{Structural VHDL Supported}

The declaration part of a structural description includes
signal declarations and component declarations.
A signal can be declared to have any of the types mentioned above. 

A component declaration must be declared with exactly the same port description
as in its entity specification.
This means that local ports are to be declared with the same name, type,
kind, and in the same order as in the entity specification.

A structural description is a set of component instantiation statements.
The ports of the instance are connected to each other through other signals
through a port map specification.
Both explicit and implicit port map specifications are supported.
The current version does not allow unconnected ports (the {\em open}
mode is not supported).

Finally, only the catenation operator (\&) can be used in the actual pat
(effective signal connected to a formal port) in a port map specification.

Note that the {\em generate} statement is not currently supported
(unfortunately!).

\subsection{Behavioral VHDL Supported}

The only type of statements supported by convert2fl are the following
concurrent statements:
\begin{enumerate}
\item
simple signal assignment
\item
conditional signal assignment
\item
selected signal assignment
\item
block statement
\end{enumerate}

When using concurrent statements, an ordinary signal can be assigned only once.
The value of the signal must be explicitly defined by the signal assignment
(for example, in a selected signal assignment, the value
of the target signal must be defined for every value that the select
expression can take on).

The above constraint is often too harsh when designing hardware
that have their control distributed (e.g., precharged lines, distributed
multiplexors, busses, etc.).
To remedy this, VHDL uses guarded-resolved signals.
A resolved signal is a signal declared with a resolved subtype.
A resolved subtype is a type together with a resolution function.
A resolved signal can be assigned by multiple signal assignments.
Depending on the value of each driver, the resolution function determines
the effective value of the signal.

A guarded signal is a resolved signal with drivers that can be disconnected.
A guarded signal must be assigned inside a {\em block} statement
through a {\em guarded} signal assignment.

To illustrate this, consider the following example of a distributed multiplexor:
\begin{hol}
signal DistributedMux : mux\_bit bus;

begin
    FirstDriver: block (Sel1 = '1' )
    begin
        DistributedMux <= guarded Data1;
    end block;

    SecondDriver: block (Sel2 = '1' )
    begin
        DistributedMux <= guarded Data2;
    end block;
end
\end{hol}

Sequential elements must be explicitly declared using he type
{\em reg\_bit} or {\em reg\_vector} (and must be of kind {\em register}).
A sequential element must be assigned inside a {\em block} statement
by a {\em guarded} signal assignment.
For example, a falling edge triggered D flip-flop could be defined as:
\begin{hol}
signal Reg : reg\_bit register;

begin
    flip\_flop: block (ck = '0' and not ck'STABLE)
    begin
        Reg <= guarded Din;
    end block;
end;
\end{hol}
On the other hand, a rising edge triggered D flip-flop with asynchronous
reset (active low) may be defined as:
\begin{hol}

signal Reg : reg\_bit register;

begin
    flip\_flop : block ((ck = '0' and not ck'stable) or (resetb = '0'))
    begin
        Reg <= guarded (resetb and Din);
    end block;
\end{hol}

Finally, level sensitive latch can be defined as:
\begin{hol}

signal Reg : reg\_bit register;

begin
    latch : block (ck = '1')
    begin
        Reg <= guarded Din;
    end block;
\end{hol}

The subset of VHDL supported by convert2fl includes only the following
built-in VHDL operators: {\bf not, and, or, xor, nor, nand, \&, =, /=}.
These operators can be applied on all types supported.
Note that other standard VHDL operators (most notably the arithmetic
and comparison operators) are not supported in this release.
