\section{.ntk Format}


An .ntk file consists of a series of commands, each of which
begins with a key character and terminates with a semicolon
`;'.  A period `.' terminates the circuit declaration.  The
key character determines how the command is interpreted.
Elements of a command are separated by spacechar characters
(see below).  Note that there must even be space before the
terminating semicolon.  All names and key characters are
case sensitive.

An .ntk file declares a circuit as a set of nodes, transistors,
and vectors.  A node is an electrical node of
type input or storage.  A transistor is a MOSFET with gate,
source and drain nodes. 


In the following description, items enclosed in braces \{ \}
may be repeated any number of times (including  0).  Items
enclosed in brackets [ ] are optional.  When there is a list
separated by vertical bars |, any item from this list may
appear.  Parentheses ( ) indicate grouping.

The following syntactic elements are referred to in the
document.

\begin{description}
\item[spacechar]
A space character.  These are: blank, tab, new-line,
carriage-return, and new-page.

\item[noderef]
A reference to a node.  Circuit nodes are numbered from
1 up to the number of nodes.  A reference to the ith
node is of the form \#i.  No node may be referenced
before it is defined.

\item[attrs]
An attribute list.  This is a sequence of the form
/char value, where char is a single character attribute
identifier, and value is the attribute value.
\end{description}

The components of the file are:

\begin{description}
\item[{(i|+|-) \{ name \} [ noderef ] ; [ attrs ; ]}]
Defines an input node.  Node type `i' denotes an ordinary
(e.g., data or clock) input node.  Node types `+'
and `-' denote power and ground nodes, respectively.
The optional list of names declares a set of names for
the node.  All node names in the circuit must be
unique.  The noderef serves only as documentation.  It
must refer to the node being defined.  The optional
attribute list defines additional properties of the
node.  Currently, the only node attribute recognized is
/c, followed by the node capacitance in picofarads
(using the C syntax for floats.)


\item[{( s | S ) size \{ name \} [ noderef ] ; [ attrs ; ]}]
Defines a storage node.  Node type `s' denotes an ordinary
storage node.  Ntk2exe may optimize such a node
out of the network, unless it is the gate of a transistor,
occurs in a vector, is an input or output to a
functional block, or can affect the value on some other
node.  Node type S denotes a visible storage node.
Such a node cannot be optimized away under any condition. 
The node size is a nonnegative integer (typically small) specifying the
node's precedence when
sharing charge with other nodes.  Node size 0 indicates
that the node does not store charge.  Any time such a
node is isolated, its state is set to the unknown value
X.  The optional list of names declares a set of names
for the node.  All node names in the circuit must be
unique.  The noderef serves only as documentation.  It
must refer to the node being defined.  The optional
attribute list defines additional properties of the
node.  Currently, the only node attribute recognized is
/c, followed by the node capacitance in picofarads
(using the C syntax for floats.)

\item[{drainref ; [ attrs ; ]( n | p | d ) [ U | Z ] [ > | < ] strength gateref sourceref}]
Defines a transistor of type `n', `p', or `d'.  The
transistor may be specified to have unit (`U') or zero
(`Z') delay.  Unit delay is the default.  The transistor
may optionally be specified to conduct information
in only one direction, either from the source to the
drain (`>') or to the source from the drain (`<').  If
no direction is specified, the transistor is bidirectional. 
The strength is a small, positive integer
specifying the transistor's precedence in ratioed circuits. 
The gate, source, and drain nodes of the
transistor are each specified by a noderef.  The
optional attribute list specifies addtional properties
of the transistor.  Currently, the only transistor
attribute recognized is /r, followed by the effective
resistance of the transistor in kilo-ohms (using the C
syntax for floats.)


\item[|  comment ;]
All text up to the terminating semicolon is ignored.
The comment string must not contain any semicolons.


\item[.]
Terminates the .ntk description.  Any following text is
ignored.  A file that does not contain a termination
command is considered incorrect.


\end{description}
