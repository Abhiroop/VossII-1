\section{.sim format}

The simulation tools crystal(1) and sim2ntk(1) accept a circuit
description in .sim format.  There is a single .sim file for the
entire circuit, unlike Magic's ext(5) format in which there is a .ext
file for every cell in a hierarchical design.

A .sim file consists of a series of lines, each of which
begins with a key letter.  The key letter beginning a line
determines how the remainder of the line is interpreted.
The following are the list of key letters understood.

\begin{description}
\item[| units: s  tech: tech]
If present, this must be the first line in the .sim
file.  It identifies the technology of this circuit as
tech and gives a scale factor for units of linear
dimension as s.  All linear dimensions appearing in the
.sim file are multiplied by s to give centimicrons.
Sim2ntk ignores the technology identifier.

\item[type  g  s  d  l  w  x  y  g=gattrs  s=sattrs  d=dattrs]
Defines a transistor of type type.  Currently, type may
be e or d for NMOS, or p or n for CMOS.  The name of
the node to which the gate, source, and drain of the
transistor are connected are given by g, s, and d
respectively.  The length and width of the transistor
are l and w.  The next two tokens, x and y, are
optional.  If present, they give the location of a
point inside the gate region of the transistor.  The
last three tokens are the attribute lists for the
transistor gate, source, and drain.  If no attributes
are present for a particular terminal, the corresponding
attribute list may be absent (i.e, there may be no
g= field at all).  The attribute lists gattrs, etc. are
comma-separated lists of labels.  The label names
should not include any spaces, although some tools can
accept label names with spaces if they are enclosed in
double quotes.  Sim2ntk(1) documents the transistor
attributes recognized by sim2ntk.

\item[C n1 n2 cap]
Defines a capacitor between nodes n1 and n2.  The value
of the capacitor is cap femtofarads.  NOTE: since many
analysis tools compute transistor gate capacitance
themselves from the transistor's area and perimeter,
the capacitance between a node and substrate (GND!)
normally does not include the capacitance from transistor
gates connected to that node.  If the .sim file was
produced by ext2sim(1), check the technology file that
was used to produce the original .ext files to see
whether transistor gate capacitance is included or
excluded; see ``Magic Maintainer's Manual 2: The Technology
File'' for details.  Sim2ntk only considers
capacitors where one terminal is ground.  It computes
transistor gate capacitances based on the transistor
constructs.

\item[A node attrs]
Associates the set of attributes attr for node node.
The attributes are specified by a comma-separated list
of strings containing no blanks.  Sim2ntk(1) documents
the node attributes recognized by sim2ntk.

\item[= node1 node2]
Each node in a .sim file is named implicitly by having
it appear in a transistor definition.  All node names
appearing in a .sim file are assumed to be distinct.
Some tools, such as esim(1) (and sim2ntk), recognize
aliases for node names.  The = construct allows the
name node2 to be defined as an alias for the name
node1.  Aliases defined by means of this construct may
not appear anywhere else in the .sim file.

\item[\verb!@! filename]
Redirects input to the specified file.  When the end-of-file
is reached, input reverts to the current file at the following line. 
Input redirection can be nested.

\end{description}
